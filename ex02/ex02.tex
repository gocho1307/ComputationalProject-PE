\documentclass[12pt,a4paper]{article}
\usepackage[legalpaper, portrait, margin=2cm]{geometry}
\usepackage[portuguese]{babel}
\usepackage[utf8]{inputenc}
\usepackage{fancyhdr}
\usepackage{graphicx}
\usepackage{hyperref}
\usepackage{indentfirst}
\usepackage{listings}
\usepackage{xcolor}

\graphicspath{ {./} }
\hypersetup{
  colorlinks=true,
  linkcolor=blue,
  filecolor=magenta,
  urlcolor=blue,
  citecolor=blue,
  pdftitle={Exercício 2 - Projeto Computacional PE 2022/2023 LEIC-A},
  pdfpagemode=FullScreen,
}

\pagenumbering{gobble}
\pagestyle{fancy}
\fancyhf{}
\rhead{Grupo \textbf{11}}
\lhead{Exercício 2 - Projeto Computacional PE 2022/2023 LEIC-A}
\cfoot{Gonçalo Bárias (103124), Raquel Braunschweig (102624) e Vasco Paisana (102533)}

\renewcommand{\footrulewidth}{0.2pt}

\renewcommand{\labelitemii}{$\circ$}
\renewcommand{\labelitemiii}{$\diamond$}

\definecolor{codegreen}{rgb}{0,0.6,0}
\definecolor{codegray}{rgb}{0.3,0.3,0.3}
\definecolor{codepurple}{rgb}{0.58,0,0.82}

\lstdefinestyle{mystyle}{
  commentstyle=\color{codegreen},
  numberstyle=\tiny\color{codegray},
  % keywordstyle=\color{magenta},
  % stringstyle=\color{codepurple},
  basicstyle=\ttfamily\footnotesize,
  breakatwhitespace=false,
  breaklines=true,
  captionpos=b,
  keepspaces=true,
  numbers=left,
  numbersep=5pt,
  showspaces=false,
  showstringspaces=false,
  showtabs=false,
  tabsize=2
}
\lstset{style=mystyle}

\begin{document}

% Introdução sobre o exercício em questão
O objetivo deste exercício é o de comparar, através de dois diagramas de extremos e quantis, os tempos médios diários (em minutos)
registados para Homens e Mulheres entre os 15 e os 64 anos em duas ocupações distintas: Cuidados pessoais e Trabalho não remunerado.
Sendo que o conjunto de dados, enviados por diversos países para a \href{https://stats.oecd.org/Index.aspx?DataSetCode=TIME_USE}{OCDE}, encontram-se num ficheiro \texttt{.csv}.
Para tal, recorreu-se ao seguinte trecho de código em \texttt{R} (utilizando as bibliotecas \texttt{ggplot2} e \texttt{dplyr}):

\quad

% Código em R para resolver o exercício
\begin{lstlisting}[language=R]
library("ggplot2")
library("dplyr")

df <- read.csv("TIME_USE_24092022.csv")
names(df) <- c("country", "occupation", "sex", "time")

df_filtered <- df %>% filter(
    country != "África do Sul" & sex == "Total" &
    occupation %in% c("Cuidados pessoais", "Trabalho não remunerado")
)

ggplot(df_filtered, aes(x = occupation, y = time, fill = occupation)) +
  geom_boxplot()+
  xlab("Ocupação") +
  ylab("Tempo médio diário (minutos)") +
  ggtitle("Análise dos tempos médios em duas ocupações") +
  labs(fill = "Ocupação", subtitle = "Cuidados pessoais e Trabalho não remunerado")
\end{lstlisting}

\quad

% Gráfico para resolver o exercício
\begin{figure}[h]
  \centering
  \includegraphics[scale = 0.8]{./ex02.png}
\end{figure}

\end{document}
