\documentclass[12pt,a4paper]{article}
\usepackage[legalpaper, portrait, margin=2cm]{geometry}
\usepackage[portuguese]{babel}
\usepackage[utf8]{inputenc}
\usepackage{fancyhdr}
\usepackage{graphicx}
\usepackage{hyperref}
\usepackage{indentfirst}
\usepackage{listings}
\usepackage{xcolor}

\graphicspath{ {./} }
\hypersetup{
  colorlinks=true,
  linkcolor=blue,
  filecolor=magenta,
  urlcolor=blue,
  citecolor=blue,
  pdftitle={Exercício 3 - Projeto Computacional PE 2022/2023 LEIC-A},
  pdfpagemode=FullScreen,
}

\pagenumbering{gobble}
\pagestyle{fancy}
\fancyhf{}
\rhead{Grupo \textbf{11}}
\lhead{Exercício 3 - Projeto Computacional PE 2022/2023 LEIC-A}
\cfoot{Gonçalo Bárias (103124), Raquel Braunschweig (102624) e Vasco Paisana (102533)}

\renewcommand{\footrulewidth}{0.2pt}

\renewcommand{\labelitemii}{$\circ$}
\renewcommand{\labelitemiii}{$\diamond$}

\definecolor{codegreen}{rgb}{0,0.6,0}
\definecolor{codegray}{rgb}{0.3,0.3,0.3}
\definecolor{codepurple}{rgb}{0.58,0,0.82}

\lstdefinestyle{mystyle}{
  commentstyle=\color{codegreen},
  numberstyle=\tiny\color{codegray},
  % keywordstyle=\color{magenta},
  % stringstyle=\color{codepurple},
  basicstyle=\ttfamily\footnotesize,
  breakatwhitespace=false,
  breaklines=true,
  captionpos=b,
  keepspaces=true,
  numbers=left,
  numbersep=5pt,
  showspaces=false,
  showstringspaces=false,
  showtabs=false,
  tabsize=2
}
\lstset{style=mystyle}

\begin{document}

% Introdução sobre o exercício em questão
O objetivo deste exercício é o de representar, através de um gráfico de barras, o \textit{Unemployment rate, by sex and age group}
entre homens e mulheres nos grupos etários 15-24, 25-54 e 55-64 durante 2015 no Reino Unido.
Sendo que o conjunto de dados, enviados por diversos países para a \href{https://stats.oecd.org/Index.aspx?DataSetCode=TIME_USE}{OCDE}, encontram-se num ficheiro \texttt{.txt}.
Para tal, recorreu-se ao seguinte trecho de código em \texttt{R} (utilizando as bibliotecas \texttt{ggplot2} e \texttt{dplyr}):

\quad

% Código em R para resolver o exercício
\begin{lstlisting}[language=R]
library("ggplot2")
library("dplyr")

df <- read.table(
  "GENDER_EMP_19032023152556091.txt", sep = "\t", header = TRUE,
  comment.char = "", na.strings = c(""," ", NA), quote = ""
)

df_filtered <- df %>% filter(
    Time == 2015 & IND == "EMP3" &
    Country == "United Kingdom" &
    Age.Group %in% c("15-24", "25-54", "55-64") &
    Sex %in% c("Men", "Women")
)

ggplot(df_filtered, aes(x = Age.Group, y = Value, fill = factor(Sex))) +
  geom_col(position = "dodge") +
  geom_text(
    aes(label = Value), vjust = 1.5,
    colour = "black", size = 6,
    position = position_dodge(.9)
  ) +
  xlab("Age Group") +
  ylab("Unemployment Rate") +
  ggtitle("Unemployment Rate, by Sex and Age Group during 2015") +
  labs(fill = "Sex", subtitle = "in the United Kingdom")
\end{lstlisting}

\quad

% Gráfico para resolver o exercício
\begin{figure}[h]
  \centering
  \includegraphics[scale = 0.8]{./ex03.png}
\end{figure}

\end{document}
